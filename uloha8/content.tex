% Hlavicka pro protokoly z fyzikalniho praktika.
% Verze pro: LaTeX
% Verze hlavicky: 22. 2. 2007
% Autor: Ustav fyziky kondenzovanych latek
% Ke stazeni: www.physics.muni.cz/ufkl/Vyuka/
% Licence: volne k pouziti, nejlepe k vcasnemu odevzdani protokolu z Vaseho mereni.

\documentclass[a4paper,11pt]{article}

% Kodovani (cestiny) v dokumentu: cp1250
% \usepackage[cp1250]{inputenc}	% Omezena stredoevropska kodova stranka, pouze MSW.
\usepackage[utf8]{inputenc}	% Doporucujeme pouzivat UTF-8 (unicode).
\usepackage{subfig}
\usepackage{float}
\usepackage{siunitx}

\usepackage{xparse}

\newcommand{\Tau}{\mathcal{T}}

\NewDocumentCommand{\codeword}{v}{%
\texttt{\textcolor{codepurple}{#1}}%
}

%%% Nemente:
\usepackage[margin=2cm]{geometry}
\newtoks\jmenopraktika \newtoks\jmeno \newtoks\datum
\newtoks\obor \newtoks\skupina \newtoks\rocnik \newtoks\semestr
\newtoks\cisloulohy \newtoks\jmenoulohy
\newtoks\tlak \newtoks\teplota \newtoks\vlhkost
%%% Nemente - konec.


%%%%%%%%%%% Doplnte pozadovane polozky:

\jmenopraktika={Fyzikální praktikum 1}  % nahradte jmenem vaseho predmetu
\jmeno={Milan Suk}            % nahradte jmenem mericiho
\datum={10. dubna 2018}        % nahradte datem mereni ulohy
\obor={F}                     % nahradte zkratkou vami studovaneho oboru
\skupina={PO 8:00}            % nahradte dobou vyuky vasi seminarni skupiny
\rocnik={I}                  % nahradte rocnikem, ve kterem studujete
\semestr={II}                 % nahradte semestrem, ve kterem studujete

\cisloulohy={8}               % nahradte cislem merene ulohy
\jmenoulohy={Měření teploty} % nahradte jmenem merene ulohy

\tlak={97,9}                   % nahradte tlakem pri mereni (v hPa)
\teplota={21,4}               % nahradte teplotou pri mereni (ve stupnich Celsia)
\vlhkost={40}               % nahradte vlhkosti vzduchu pri mereni (v %)

%%%%%%%%%%% Konec pozadovanych polozek.


%%%%%%%%%%% Uzitecne balicky:
\usepackage[czech]{babel}
\usepackage{graphicx}
\usepackage{amsmath}
\usepackage{xspace}
\usepackage{url}
\usepackage{indentfirst}
\usepackage{listings}
\usepackage{color}


\definecolor{codegreen}{rgb}{0,0.6,0}
\definecolor{codegray}{rgb}{0.5,0.5,0.5}
\definecolor{codepurple}{rgb}{0.58,0,0.82}
\definecolor{backcolour}{rgb}{0.95,0.95,0.92}
 
\lstdefinestyle{mystyle}{
    backgroundcolor=\color{backcolour},   
    commentstyle=\color{codegreen},
    keywordstyle=\color{magenta},
    numberstyle=\tiny\color{codegray},
    stringstyle=\color{codepurple},
    basicstyle=\footnotesize,
    breakatwhitespace=false,         
    breaklines=true,                 
    captionpos=b,                    
    keepspaces=true,                 
    numbers=left,                    
    numbersep=5pt,                  
    showspaces=false,                
    showstringspaces=false,
    showtabs=false,                  
    tabsize=2
}
 
\lstset{style=mystyle}

%%%%%% Zamezeni parchantu:
\widowpenalty 10000 \clubpenalty 10000 \displaywidowpenalty 10000
%%%%%% Parametry pro moznost vsazeni vetsiho poctu obrazku na stranku
\setcounter{topnumber}{3}	  % max. pocet floatu nahore (specifikace t)
\setcounter{bottomnumber}{3}	  % max. pocet floatu dole (specifikace b)
\setcounter{totalnumber}{6}	  % max. pocet floatu na strance celkem
\renewcommand\topfraction{0.9}	  % max podil stranky pro floaty nahore
\renewcommand\bottomfraction{0.9} % max podil stranky pro floaty dole
\renewcommand\textfraction{0.1}	  % min podil stranky, ktery musi obsahovat text
\intextsep=8mm \textfloatsep=8mm  %\intextsep pro ulozeni [h] floatu a \textfloatsep pro [b] or [t]

% Tecky za cisly sekci:
\renewcommand{\thesection}{\arabic{section}.}
\renewcommand{\thesubsection}{\thesection\arabic{subsection}.}
% Jednopismenna mezera mezi cislem a nazvem kapitoly:
\makeatletter \def\@seccntformat#1{\csname the#1\endcsname\hspace{1ex}} \makeatother


%%%%%%%%%%%%%%%%%%%%%%%%%%%%%%%%%%%%%%%%%%%%%%%%%%%%%%%%%%%%%%%%%%%%%%%%%%%%%%%
%%%%%%%%%%%%%%%%%%%%%%%%%%%%%%%%%%%%%%%%%%%%%%%%%%%%%%%%%%%%%%%%%%%%%%%%%%%%%%%
% Zacatek dokumentu
%%%%%%%%%%%%%%%%%%%%%%%%%%%%%%%%%%%%%%%%%%%%%%%%%%%%%%%%%%%%%%%%%%%%%%%%%%%%%%%
%%%%%%%%%%%%%%%%%%%%%%%%%%%%%%%%%%%%%%%%%%%%%%%%%%%%%%%%%%%%%%%%%%%%%%%%%%%%%%%

\begin{document}

%%%%%%%%%%%%%%%%%%%%%%%%%%%%%%%%%%%%%%%%%%%%%%%%%%%%%%%%%%%%%%%%%%%%%%%%%%%%%%%
% Nemente:
%%%%%%%%%%%%%%%%%%%%%%%%%%%%%%%%%%%%%%%%%%%%%%%%%%%%%%%%%%%%%%%%%%%%%%%%%%%%%%%
\thispagestyle{empty}

{
\begin{center}
\sf 
{\Large Ústav fyzikální elektroniky Přírodovědecké fakulty Masarykovy univerzity} \\
\bigskip
{\huge \bfseries FYZIKÁLNÍ PRAKTIKUM} \\
\bigskip
{\Large \the\jmenopraktika}
\end{center}

\bigskip

\sf
\noindent
\setlength{\arrayrulewidth}{1pt}
\begin{tabular*}{\textwidth}{@{\extracolsep{\fill}} l l}
\large {\bfseries Zpracoval:}  \the\jmeno & \large  {\bfseries Naměřeno:} \the\datum\\[2mm]
\large  {\bfseries Obor:} \the\obor  \hspace{40mm}  {\bfseries Skupina:} \the\skupina %
&\large {\bfseries Testováno:}\\
\\
\hline
\end{tabular*}
}

\bigskip

{
\sf
\noindent \begin{tabular}{p{3cm} p{0.6\textwidth}}
\Large  Úloha č. {\bfseries \the\cisloulohy:} \par
&\Large \bfseries \the\jmenoulohy  \\[2mm]
\end{tabular}
}

%%%%%%%%%%%%%%%%%%%%%%%%%%%%%%%%%%%%%%%%%%%%%%%%%%%%%%%%%%%%%%%%%%%%%%%%%%%%%%%
% konec Nemente.
%%%%%%%%%%%%%%%%%%%%%%%%%%%%%%%%%%%%%%%%%%%%%%%%%%%%%%%%%%%%%%%%%%%%%%%%%%%%%%%

%%%%%%%%%%%%%%%%%%%%%%%%%%%%%%%%%%%%%%%%%%%%%%%%%%%%%%%%%%%%%%%%%%%%%%%%%%%%%%%
%%%%%%%%%%%%%%%%%%%%%%%%%%%%%%%%%%%%%%%%%%%%%%%%%%%%%%%%%%%%%%%%%%%%%%%%%%%%%%%
% Zacatek textu vlastniho protokolu
%%%%%%%%%%%%%%%%%%%%%%%%%%%%%%%%%%%%%%%%%%%%%%%%%%%%%%%%%%%%%%%%%%%%%%%%%%%%%%%
%%%%%%%%%%%%%%%%%%%%%%%%%%%%%%%%%%%%%%%%%%%%%%%%%%%%%%%%%%%%%%%%%%%%%%%%%%%%%%%


\section{Úvod}

    \paragraph{} V první části jsem zjišťoval závislost teploty na odporu
    v olejové lázni. Teplota se nechala vzrůstat z teploty $20^{\circ}C$ na 
    $120^{\circ}C$. Linerní regresí jsem pak  mohl určit hodnotu $\alpha$, pro 
    kterou při lineární závislosti $R = a + b \cdot t$ platí

    \begin{equation}
        \alpha = \frac{b}{a}
    \end{equation}

    \paragraph{} V druhé části jsem zjišťoval relaxační dobu odporového čidla.
    Pro relaxační dobu $\tau_{m}$ platí vztah

    \begin{equation}
        t(\tau) = t_{2} - (t_{2} - t_{1}) e^{- \frac{\tau}{\tau_{m}}}
    \end{equation}

    \paragraph{} Dále jsem proměřoval emisivitu desky pokryté černým, bílým
    a aluminiovým lakem. Deska se vyhřála na teplotu $T$ a pak se proměřila
    infračerveným teploměrem a získala se hodnota $T_{P}$. Pro emisivitu 
    $\epsilon$ pak platí

    \begin{equation}
        \epsilon = \frac{T_{P}^{4}}{T^{4}}
    \end{equation}

    \paragraph{} Ve čtvrté části jsem emisivitu měřil přes okánka z různých
    materiálů a určoval propustnost $\Tau$, pro kterou platí 

    \begin{equation}
        \Tau = \frac{T_{P}^{4}}{T^{4}}
    \end{equation}

    kde $T_{P}$ je zde teplota měřená přes okénko a $T_{P}$ bez okénka.

    \paragraph{} V poslední části jsem proměřoval emisivitu povrchu s námrazou
    a bez námrazi. K vyhodnocení lze použí rovnici (3).

    \begin{equation}
        \Tau = \frac{T_{P}^{4}}{T^{4}}
    \end{equation}
    
\section{Postup měření}

    \subsection{Teplotní závislost odporových čidel}

        \paragraph{} Lieární regresí jsem určil koeficienty lineární funkce a
        pomocí nich určil hodnotu $\alpha$.

\begin{lstlisting}[language=Bash][H]
from scipy import stats

data_raw = open('mereni_8_000.txt').readlines()
x = list()
y = list()

for line in data_raw:
    d = line.replace(',' ,'.').split('\t')
    x.append(float(d[1]))
    y.append(float(d[4]))

b, a, _, _, _ = stats.linregress(x, y)
alpha = b / a

print(f"R_0 = {a}")
print(f"alpha = {alpha}")\end{lstlisting}

        výsledek vyhodnocení je

        $$ R_{0} = 954.42 \, \si{\ohm} $$
        $$ \alpha = 0.00679 \, K^{-1} $$

    \subsection{Relaxační doba odporového čidla}

        \paragraph{} Nasbíraná data jsem nafitoval na funkci $a + b \cdot e^{- \frac{x}{c}}$.
        Parameter $c$ pak přímo odpovídá hledané relaxační době.

\begin{lstlisting}[language=Bash][H]
from numpy import exp
from scipy.optimize import curve_fit

data_raw = open('suk_mereni_000.txt').readlines()
x = list()
y = list()

for line in data_raw:
    d = line.replace(',' ,'.').split('\t')
    x.append(float(d[0]))
    y.append(float(d[1]))

def func(x, a, b, c):
    return a + b * exp(- x / c)

popt, pcov = curve_fit(func, x, y)

print(popt)\end{lstlisting}

        $$ \tau_{m} = \left(89 \pm 8\right) \, s$$

    \subsection{Emisivita teploměru}

        \paragraph{} Nejdříve jsem určil střední hodnotu teploty určené infračerveným
        teploměrem, pak jsem dosadil do rovnice pro emisivitu.

\begin{lstlisting}[language=Bash][H]
import numpy
from math import pow, sqrt

pow2 = lambda x: x * x

T = 483.15
uT = 0.5

data_raw = open('data1.txt').readlines()
data = list()
data_sub = list()

for line in data_raw:
    if line.strip() == '':
        data.append(data_sub)
        data_sub = []
    else:
        data_sub.append(float(line))

def calc_epsilon(d: list):
    t_p = numpy.average(d)

    epsilon = pow(t_p / T, 4)
    epsilon_err = epsilon * sqrt(pow2(4 * uT / T))

    return (epsilon, epsilon_err)

for d in data:
    print(calc_epsilon(d))\end{lstlisting}

    Pro jednotlivé materiály pak emisivita vychází

    $$ \epsilon_{cerna} = \left(0.991 \pm 0.001\right)$$
    $$ \epsilon_{bila} = \left(1.400 \pm 0.001\right)$$
    $$ \epsilon_{aluminium} = \left(0.3847 \pm 0.0003\right)$$

    \subsection{Propustnost $\tau$ okének}

        \paragraph{} Co se týče analýzy tohoto experimentu, jedná se ve skutečnosti
        o téměř stejný postup jako u předchozí části, takže na vyhodnocení výsledků
        jsem použil úplně stejný script (naštěstí je napsán dostatečně oebcně, aby s ním
        šlo vyhodnotit libovolný počet měření v jednom kroku), jen jsem místo skutečné
        teploty $T$ používal teplotu naměřenou infračerveným teploměrem bez okénka.

        $$ \Tau_{Cu} = 0.13 $$
        $$ \Tau_{GaAs} = 0.46 $$
        $$ \Tau_{Si} = 0.44 $$
        $$ \Tau_{NaCl} = 0.55 $$
        $$ \Tau_{KBr} = 0.20 $$
        $$ \Tau_{CaF2} = 0.16 $$
        $$ \Tau_{SiO2} = 0.15 $$
        $$ \Tau_{polykarbonat} = 0.16 $$

    \subsection{Emisivita povrchu s námrazou}

        \paragraph{} K vyhodnocení tohot měření jsem upět použil variace na script
        z předchozích měření. Spočítal jsem střední hodnoty teplot změřených kontaktním
        teploměrem a pak infračerveným. Pak jsem jsem vzhledem k těmto hodnatám určil 
        emisivitu podle rovnice (4).

\begin{lstlisting}[language=Bash][H]
import numpy
from math import pow, sqrt

pow2 = lambda x: x * x

data_raw = open('data3.txt').readlines()
data = list()
data_sub = list()

for line in data_raw:
    if line.strip() == '':
        data.append(data_sub)
        data_sub = []
    else:
        data_sub.append(float(line) + 273.15)

def calc_epsilon(t1_l: list, t2_l: list):
    t_1 = numpy.average(t1_l)
    t_1_err = numpy.average((t1_l - t_1)**2)
    t_2 = numpy.average(t2_l)
    t_2_err = numpy.average((t1_l - t_1)**2)

    epsilon = pow(t_1 / t_2, 4)
    epsilon_err = epsilon * sqrt(pow2(4 * t_1_err / t_1) + pow2(4 * t_2_err / t_2))
    return (epsilon, epsilon_err)

print(calc_epsilon(data[0], data[1]))
print(calc_epsilon(data[2], data[3]))\end{lstlisting}

    $$ \epsilon_{1} = 0.826 \pm 0.007 $$
    $$ \epsilon_{2} = 0.937 \pm 0.008 $$

    kde $\epsilon_{1}$ je emisivita povrchu s námrazou a $\epsilon_{2}$ 
    povrchu bez námrazy.

\section{Výsledky}

    \paragraph{} Ve srovnáním s tabulkovými hodnotami jsou moje výsledky minimálně řádově
    v pořádku. Trochu překvapující jsou pro mě hodnoty emisivity povrchu s námrazou a bez
    námrazy, kde jsem očekával větší rozdíly těchto dvou hodnot. Zajímavé při měření bylo,
    že jednotlivá měření pomocí infračerveného teploměru měla relativně velké roztyly, proto
    se u některých měření špatně určovaly odchylky hodnot, protože samotné rozptyly teplot byly
    řádově větší, než hodnoty emisivity samotné.
    
\end{document}

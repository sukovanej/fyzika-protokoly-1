% Hlavicka pro protokoly z fyzikalniho praktika.
% Verze pro: LaTeX
% Verze hlavicky: 22. 2. 2007
% Autor: Ustav fyziky kondenzovanych latek
% Ke stazeni: www.physics.muni.cz/ufkl/Vyuka/
% Licence: volne k pouziti, nejlepe k vcasnemu odevzdani protokolu z Vaseho mereni.

\documentclass[a4paper,11pt]{article}

% Kodovani (cestiny) v dokumentu: cp1250
% \usepackage[cp1250]{inputenc}	% Omezena stredoevropska kodova stranka, pouze MSW.
\usepackage[utf8]{inputenc}	% Doporucujeme pouzivat UTF-8 (unicode).
\usepackage[demo]{graphicx}

%%% Nemente:
\usepackage[margin=2cm]{geometry}
\newtoks\jmenopraktika \newtoks\jmeno \newtoks\datum
\newtoks\obor \newtoks\skupina \newtoks\rocnik \newtoks\semestr
\newtoks\cisloulohy \newtoks\jmenoulohy
\newtoks\tlak \newtoks\teplota \newtoks\vlhkost
%%% Nemente - konec.


%%%%%%%%%%% Doplnte pozadovane polozky:

\jmenopraktika={Fyzikální praktikum 1}  % nahradte jmenem vaseho predmetu
\jmeno={Milan Suk}            % nahradte jmenem mericiho
\datum={26. února 2018}        % nahradte datem mereni ulohy
\obor={F}                     % nahradte zkratkou vami studovaneho oboru
\skupina={PO 8:00}            % nahradte dobou vyuky vasi seminarni skupiny
\rocnik={I}                  % nahradte rocnikem, ve kterem studujete
\semestr={II}                 % nahradte semestrem, ve kterem studujete

\cisloulohy={3}               % nahradte cislem merene ulohy
\jmenoulohy={Měření viskozity, hustoty a povrchového napětí kapalin} % nahradte jmenem merene ulohy

\tlak={97,9}                   % nahradte tlakem pri mereni (v hPa)
\teplota={21,4}               % nahradte teplotou pri mereni (ve stupnich Celsia)
\vlhkost={40}               % nahradte vlhkosti vzduchu pri mereni (v %)

%%%%%%%%%%% Konec pozadovanych polozek.


%%%%%%%%%%% Uzitecne balicky:
\usepackage[czech]{babel}
\usepackage{graphicx}
\usepackage{amsmath}
\usepackage{xspace}
\usepackage{url}
\usepackage{indentfirst}
\usepackage{listings}
\usepackage{color}


\definecolor{codegreen}{rgb}{0,0.6,0}
\definecolor{codegray}{rgb}{0.5,0.5,0.5}
\definecolor{codepurple}{rgb}{0.58,0,0.82}
\definecolor{backcolour}{rgb}{0.95,0.95,0.92}
 
\lstdefinestyle{mystyle}{
    backgroundcolor=\color{backcolour},   
    commentstyle=\color{codegreen},
    keywordstyle=\color{magenta},
    numberstyle=\tiny\color{codegray},
    stringstyle=\color{codepurple},
    basicstyle=\footnotesize,
    breakatwhitespace=false,         
    breaklines=true,                 
    captionpos=b,                    
    keepspaces=true,                 
    numbers=left,                    
    numbersep=5pt,                  
    showspaces=false,                
    showstringspaces=false,
    showtabs=false,                  
    tabsize=2
}
 
\lstset{style=mystyle}

%%%%%% Zamezeni parchantu:
\widowpenalty 10000 \clubpenalty 10000 \displaywidowpenalty 10000
%%%%%% Parametry pro moznost vsazeni vetsiho poctu obrazku na stranku
\setcounter{topnumber}{3}	  % max. pocet floatu nahore (specifikace t)
\setcounter{bottomnumber}{3}	  % max. pocet floatu dole (specifikace b)
\setcounter{totalnumber}{6}	  % max. pocet floatu na strance celkem
\renewcommand\topfraction{0.9}	  % max podil stranky pro floaty nahore
\renewcommand\bottomfraction{0.9} % max podil stranky pro floaty dole
\renewcommand\textfraction{0.1}	  % min podil stranky, ktery musi obsahovat text
\intextsep=8mm \textfloatsep=8mm  %\intextsep pro ulozeni [h] floatu a \textfloatsep pro [b] or [t]

% Tecky za cisly sekci:
\renewcommand{\thesection}{\arabic{section}.}
\renewcommand{\thesubsection}{\thesection\arabic{subsection}.}
% Jednopismenna mezera mezi cislem a nazvem kapitoly:
\makeatletter \def\@seccntformat#1{\csname the#1\endcsname\hspace{1ex}} \makeatother


%%%%%%%%%%%%%%%%%%%%%%%%%%%%%%%%%%%%%%%%%%%%%%%%%%%%%%%%%%%%%%%%%%%%%%%%%%%%%%%
%%%%%%%%%%%%%%%%%%%%%%%%%%%%%%%%%%%%%%%%%%%%%%%%%%%%%%%%%%%%%%%%%%%%%%%%%%%%%%%
% Zacatek dokumentu
%%%%%%%%%%%%%%%%%%%%%%%%%%%%%%%%%%%%%%%%%%%%%%%%%%%%%%%%%%%%%%%%%%%%%%%%%%%%%%%
%%%%%%%%%%%%%%%%%%%%%%%%%%%%%%%%%%%%%%%%%%%%%%%%%%%%%%%%%%%%%%%%%%%%%%%%%%%%%%%

\begin{document}

%%%%%%%%%%%%%%%%%%%%%%%%%%%%%%%%%%%%%%%%%%%%%%%%%%%%%%%%%%%%%%%%%%%%%%%%%%%%%%%
% Nemente:
%%%%%%%%%%%%%%%%%%%%%%%%%%%%%%%%%%%%%%%%%%%%%%%%%%%%%%%%%%%%%%%%%%%%%%%%%%%%%%%
\thispagestyle{empty}

{
\begin{center}
\sf 
{\Large Ústav fyzikální elektroniky Přírodovědecké fakulty Masarykovy univerzity} \\
\bigskip
{\huge \bfseries FYZIKÁLNÍ PRAKTIKUM} \\
\bigskip
{\Large \the\jmenopraktika}
\end{center}

\bigskip

\sf
\noindent
\setlength{\arrayrulewidth}{1pt}
\begin{tabular*}{\textwidth}{@{\extracolsep{\fill}} l l}
\large {\bfseries Zpracoval:}  \the\jmeno & \large  {\bfseries Naměřeno:} \the\datum\\[2mm]
\large  {\bfseries Obor:} \the\obor  \hspace{40mm}  {\bfseries Skupina:} \the\skupina %
&\large {\bfseries Testováno:}\\
\\
\hline
\end{tabular*}
}

\bigskip

{
\sf
\noindent \begin{tabular}{p{3cm} p{0.6\textwidth}}
\Large  Úloha č. {\bfseries \the\cisloulohy:} \par
&\Large \bfseries \the\jmenoulohy  \\[2mm]
\end{tabular}
}

%%%%%%%%%%%%%%%%%%%%%%%%%%%%%%%%%%%%%%%%%%%%%%%%%%%%%%%%%%%%%%%%%%%%%%%%%%%%%%%
% konec Nemente.
%%%%%%%%%%%%%%%%%%%%%%%%%%%%%%%%%%%%%%%%%%%%%%%%%%%%%%%%%%%%%%%%%%%%%%%%%%%%%%%

%%%%%%%%%%%%%%%%%%%%%%%%%%%%%%%%%%%%%%%%%%%%%%%%%%%%%%%%%%%%%%%%%%%%%%%%%%%%%%%
%%%%%%%%%%%%%%%%%%%%%%%%%%%%%%%%%%%%%%%%%%%%%%%%%%%%%%%%%%%%%%%%%%%%%%%%%%%%%%%
% Zacatek textu vlastniho protokolu
%%%%%%%%%%%%%%%%%%%%%%%%%%%%%%%%%%%%%%%%%%%%%%%%%%%%%%%%%%%%%%%%%%%%%%%%%%%%%%%
%%%%%%%%%%%%%%%%%%%%%%%%%%%%%%%%%%%%%%%%%%%%%%%%%%%%%%%%%%%%%%%%%%%%%%%%%%%%%%%


\section{Úvod}

    \paragraph{} V tomto měření se měla měřenít vizkozita, hustota a povrchové
    napětí kapalin. V první části jsem měřil kinematickou viskozitu vody pro 
    více teplot pomocí \textbf{Ubbhelohdeho viskozimetru}. V tomto měření se 
    voda unitř viskozimetru ohrála na požadovanou teplotu umístěním do nádoby
    s vodou požadované teploty. Následně se měřil čas poklesu hladiny z horní 
    rysky na dolní. Ze vztahu

    
    \begin{equation}
        \nu = K \cdot t
    \end{equation}

    se následně určila hledaná viskozita.

    \paragraph{} V dalším měření se vizkosita určila pomocí výtoku vody z 
    \textbf{Mariottovy láhve}. Změřil se čas, po který voda vytékala, a změna tlaku 
    $p = p_{2} - p_{1}$. Viskozita se potom určí pomocí následující rovnice.

    \begin{equation}
        \eta = \frac{\pi R^{4} p t}{8 V L}
    \end{equation}

    kde $V$ je celkový objem, který vyteče a $L$ délka kapiláry. 

    \paragraph{} Poté jsem určoval hustotu lihu pomocí \textbf{pyknometru} a známé
    hustoty vody. Pro neznámou hustotu lihu platí

    \begin{equation}
        \rho = (\rho_{voda} - \rho_{vzduch}) \frac{m - m_{0}}{m_{voda} - m_{0}} 
        + \rho_{vzduch}
    \end{equation}

    kde $m_{0}$ je hmotnost prázdného pyknometru a $m$ hmotnost pyknometru s lihem.

    \paragraph{} Hustotu lihu jsem pak ještě určil pomocí medoty \textbf{ponorného
    tělíska}.  Měření probíhalo na vahách s délním zavěšením. Váhy se vytárovaly na 
    hmotnost tělíska, pak jsem změřila hmotnost $m$ při ponoření tělíska do nádoby 
    s lihem a potom $m_{voda}$ při ponoření do nádoby s vodou. Hustota lze pak spočítat
    pomocí vztahu

    \begin{equation}
        \rho = \frac{m}{m_{voda}} \rho_{voda}
    \end{equation}

    \paragraph{} Dále se mělo změřit povrchové napětí vody pomocí \textbf{du Noüyho 
    metody kroužku}. Při tomto měření se určila největší síla působící na kroužek,
    aby se pověrchové napětí určitě dle vztahu

    \begin{equation}
        \sigma = \frac{F_{max}}{4 \pi R} \cdot f
    \end{equation}

    kde $f$ je Harkinsův-Jordanův korekční faktor.

    \paragraph{} V posledním měření jsem určoval disporzní složku povrchové
    energie vody metodou kontaktního úhlu. Jako kalibrační kapalinu jsem použil 
    glycerol a metylen jodid.

    \begin{equation}
        \frac{\sigma^{lw}}{\sigma} = \frac{\sigma^{lw}_{kal}}{\sigma_{kal}}
        \frac{\sigma}{\sigma_{kal}} \left(\frac{1 + \cos \theta}{1 + \cos \theta_{kal}}\right)^{2}
    \end{equation}

\section{Výsledky}

    \subsection{Měření viskozity vody}

        \paragraph{} Prováděl jsem měření pro dvě teploty ($20 ^{\circ}C$ a $30 ^{\circ}C$). Získané časy jsou

        $$ t_{20} = 13:56:46 \\ $$
        $$ t_{30} = 11:36:16 $$

        odtud jsem zjistil, že viskozita vody pro jednotlivé časy by měla být

        $$ \nu_{20} = 8.887 \cdot 10^{-7} \, m^{2} \cdot s^{-1} \\ $$
        $$ \nu_{30} = 7.398 \cdot 10^{-7} \, m^{2} \cdot s^{-1} $$

    \subsection{Absolutní měření viskozity}

        \paragraph{} Při této mětodě jsem měřil čas vytíkání vody kapilarou z
        Mariottovy láhve. Délka kapiláry je $L = \left(165.0 \pm 0.5\right) \, mm$ a její průměr
        $R = \left(0.570 \pm 0.001\right) \, mm$. Pak jsem zjišťoval změnu tlaku tak, že jsem změřil změnu 
        výšku hladiny, určit jsem objem vody, která vytekla, a čas trvání tohoto procesu.

        $$t = 108 \, s$$
        $$\Delta V = 2.82 \cdot 10^{-5} m^{3} $$
        $$p = 1096.4 \, Pa $$

        Po dosazení vychází výsledná viskozita

        $$ \eta = \left(1.055 \pm 0.003\right) \, Pa \cdot s$$

    \subsection{Pyknometrická metoda}

        \paragraph{} Změřil jsem jednotlivé hmotnosti

        $$ m_{1} = 42.3 \, g $$
        $$ m_{2} = 44.4 \, g $$
        $$ m_{1} = 23.5 \, g $$

        \paragraph{} po dosazení do vztahu (3) dostávám

        $$ \rho_{lih} = 897.0 \pm 0.1 \, kg \cdot m^{-3} $$

    \subsection{Ponorné tělísko}

        \paragraph{} V tomto měření hustoty lihu jsem nejdříve změřil hmotnost $m_{1}$ s
        destilovanou vodou a potom $m_{2}$ s lihem.

        $$ m_{1} = 5.76 \, g $$
        $$ m_{2} = 11.30 \, g $$

        \paragraph{} a po dosazení do rovnice (4) dostávám

        $$ \rho_{lih} = 508.21 \pm 0.01 \, kg \cdot m^{-3} $$
    
    \subsection{du Nouyho metoda kroužku}

        \paragraph{} Při této mětodě jsem určoval maxinální sílu působící mezi kroužkem a
        hladinou kapaliny. Skriptem níže jsem zanalyzoval data (jednoduše jsem našel největší
        hodnotu v seznamu s hodnotamy síly)

\begin{lstlisting}[language=Python][h]
import numpy
data = numpy.loadtxt(file_name)
x = [i[0] for i in data]
print(max(y))\end{lstlisting}

        \paragraph{} Z těchto hodnot jsem určit hodnotu průměrnou

        $$ F_{max} = \left(35.38 \pm 0.01 \right) \cdot 10^{-3} \, N $$

        a výsledné povrchové napětí je

        $$ \sigma = \left(74.8 \pm 0.3 \right) \cdot 10^{-3} \, N \cdot m^{-1} $$

    \subsection{Metoda kontaktního úhlu}

        \paragraph{} Při této metodě jsem měřil kontaktní úhel vody z naměřených kontaktních 
        úhlů a známých napětí kalibračních kapalin (glycerol a metylen jodid) pomocí vztahu 
        (6).
        
        $$\theta = 96.8 ^{\circ}$$
        $$\theta_{1} = 73.1 ^{\circ}$$
        $$\theta_{2} = 91.8 ^{\circ}$$

        Z těchto hodnot jsem dopočítal následující povrchové napětí vody.

        $$\sigma_{1} = \left(16.2 \pm 0.1 \right) \cdot 10^{-3} \, N \cdot m^{-1}$$
        $$\sigma_{2} = \left(61.2 \pm 0.1 \right) \cdot 10^{-3} \, N \cdot m^{-1}$$

\section{Zhodnocení měření, závěr}

    \paragraph{} Po dokončení měření je porovnáním s tabulkovými hodnotami vidět, že výsledky
    viskozit a povrchových napětí vyšli poměrně dobře (až na odchylku $\sigma_{1}$ v posledním měření
    pomocí metylen jodidu). Mnohem lepší shodu jsem čekal u měření hustoty pomocí pyknometrické metody,
    pravděpodobně bylo měření zatíženo hrubou chybou.

\end{document}

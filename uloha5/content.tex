% Hlavicka pro protokoly z fyzikalniho praktika.
% Verze pro: LaTeX
% Verze hlavicky: 22. 2. 2007
% Autor: Ustav fyziky kondenzovanych latek
% Ke stazeni: www.physics.muni.cz/ufkl/Vyuka/
% Licence: volne k pouziti, nejlepe k vcasnemu odevzdani protokolu z Vaseho mereni.

\documentclass[a4paper,11pt]{article}

% Kodovani (cestiny) v dokumentu: cp1250
% \usepackage[cp1250]{inputenc}	% Omezena stredoevropska kodova stranka, pouze MSW.
\usepackage[utf8]{inputenc}	% Doporucujeme pouzivat UTF-8 (unicode).
\usepackage[demo]{graphicx}

%%% Nemente:
\usepackage[margin=2cm]{geometry}
\newtoks\jmenopraktika \newtoks\jmeno \newtoks\datum
\newtoks\obor \newtoks\skupina \newtoks\rocnik \newtoks\semestr
\newtoks\cisloulohy \newtoks\jmenoulohy
\newtoks\tlak \newtoks\teplota \newtoks\vlhkost
%%% Nemente - konec.


%%%%%%%%%%% Doplnte pozadovane polozky:

\jmenopraktika={Fyzikální praktikum 1}  % nahradte jmenem vaseho predmetu
\jmeno={Milan Suk}            % nahradte jmenem mericiho
\datum={19. března 2018}        % nahradte datem mereni ulohy
\obor={F}                     % nahradte zkratkou vami studovaneho oboru
\skupina={PO 8:00}            % nahradte dobou vyuky vasi seminarni skupiny
\rocnik={I}                  % nahradte rocnikem, ve kterem studujete
\semestr={II}                 % nahradte semestrem, ve kterem studujete

\cisloulohy={5}               % nahradte cislem merene ulohy
\jmenoulohy={Měření modul pružnosti pevných látek} % nahradte jmenem merene ulohy

\tlak={97,9}                   % nahradte tlakem pri mereni (v hPa)
\teplota={21,4}               % nahradte teplotou pri mereni (ve stupnich Celsia)
\vlhkost={40}               % nahradte vlhkosti vzduchu pri mereni (v %)

%%%%%%%%%%% Konec pozadovanych polozek.


%%%%%%%%%%% Uzitecne balicky:
\usepackage[czech]{babel}
\usepackage{graphicx}
\usepackage{amsmath}
\usepackage{xspace}
\usepackage{url}
\usepackage{indentfirst}
\usepackage{listings}
\usepackage{color}


\definecolor{codegreen}{rgb}{0,0.6,0}
\definecolor{codegray}{rgb}{0.5,0.5,0.5}
\definecolor{codepurple}{rgb}{0.58,0,0.82}
\definecolor{backcolour}{rgb}{0.95,0.95,0.92}
 
\lstdefinestyle{mystyle}{
    backgroundcolor=\color{backcolour},   
    commentstyle=\color{codegreen},
    keywordstyle=\color{magenta},
    numberstyle=\tiny\color{codegray},
    stringstyle=\color{codepurple},
    basicstyle=\footnotesize,
    breakatwhitespace=false,         
    breaklines=true,                 
    captionpos=b,                    
    keepspaces=true,                 
    numbers=left,                    
    numbersep=5pt,                  
    showspaces=false,                
    showstringspaces=false,
    showtabs=false,                  
    tabsize=2
}
 
\lstset{style=mystyle}

%%%%%% Zamezeni parchantu:
\widowpenalty 10000 \clubpenalty 10000 \displaywidowpenalty 10000
%%%%%% Parametry pro moznost vsazeni vetsiho poctu obrazku na stranku
\setcounter{topnumber}{3}	  % max. pocet floatu nahore (specifikace t)
\setcounter{bottomnumber}{3}	  % max. pocet floatu dole (specifikace b)
\setcounter{totalnumber}{6}	  % max. pocet floatu na strance celkem
\renewcommand\topfraction{0.9}	  % max podil stranky pro floaty nahore
\renewcommand\bottomfraction{0.9} % max podil stranky pro floaty dole
\renewcommand\textfraction{0.1}	  % min podil stranky, ktery musi obsahovat text
\intextsep=8mm \textfloatsep=8mm  %\intextsep pro ulozeni [h] floatu a \textfloatsep pro [b] or [t]

% Tecky za cisly sekci:
\renewcommand{\thesection}{\arabic{section}.}
\renewcommand{\thesubsection}{\thesection\arabic{subsection}.}
% Jednopismenna mezera mezi cislem a nazvem kapitoly:
\makeatletter \def\@seccntformat#1{\csname the#1\endcsname\hspace{1ex}} \makeatother


%%%%%%%%%%%%%%%%%%%%%%%%%%%%%%%%%%%%%%%%%%%%%%%%%%%%%%%%%%%%%%%%%%%%%%%%%%%%%%%
%%%%%%%%%%%%%%%%%%%%%%%%%%%%%%%%%%%%%%%%%%%%%%%%%%%%%%%%%%%%%%%%%%%%%%%%%%%%%%%
% Zacatek dokumentu
%%%%%%%%%%%%%%%%%%%%%%%%%%%%%%%%%%%%%%%%%%%%%%%%%%%%%%%%%%%%%%%%%%%%%%%%%%%%%%%
%%%%%%%%%%%%%%%%%%%%%%%%%%%%%%%%%%%%%%%%%%%%%%%%%%%%%%%%%%%%%%%%%%%%%%%%%%%%%%%

\begin{document}

%%%%%%%%%%%%%%%%%%%%%%%%%%%%%%%%%%%%%%%%%%%%%%%%%%%%%%%%%%%%%%%%%%%%%%%%%%%%%%%
% Nemente:
%%%%%%%%%%%%%%%%%%%%%%%%%%%%%%%%%%%%%%%%%%%%%%%%%%%%%%%%%%%%%%%%%%%%%%%%%%%%%%%
\thispagestyle{empty}

{
\begin{center}
\sf 
{\Large Ústav fyzikální elektroniky Přírodovědecké fakulty Masarykovy univerzity} \\
\bigskip
{\huge \bfseries FYZIKÁLNÍ PRAKTIKUM} \\
\bigskip
{\Large \the\jmenopraktika}
\end{center}

\bigskip

\sf
\noindent
\setlength{\arrayrulewidth}{1pt}
\begin{tabular*}{\textwidth}{@{\extracolsep{\fill}} l l}
\large {\bfseries Zpracoval:}  \the\jmeno & \large  {\bfseries Naměřeno:} \the\datum\\[2mm]
\large  {\bfseries Obor:} \the\obor  \hspace{40mm}  {\bfseries Skupina:} \the\skupina %
&\large {\bfseries Testováno:}\\
\\
\hline
\end{tabular*}
}

\bigskip

{
\sf
\noindent \begin{tabular}{p{3cm} p{0.6\textwidth}}
\Large  Úloha č. {\bfseries \the\cisloulohy:} \par
&\Large \bfseries \the\jmenoulohy  \\[2mm]
\end{tabular}
}

%%%%%%%%%%%%%%%%%%%%%%%%%%%%%%%%%%%%%%%%%%%%%%%%%%%%%%%%%%%%%%%%%%%%%%%%%%%%%%%
% konec Nemente.
%%%%%%%%%%%%%%%%%%%%%%%%%%%%%%%%%%%%%%%%%%%%%%%%%%%%%%%%%%%%%%%%%%%%%%%%%%%%%%%

%%%%%%%%%%%%%%%%%%%%%%%%%%%%%%%%%%%%%%%%%%%%%%%%%%%%%%%%%%%%%%%%%%%%%%%%%%%%%%%
%%%%%%%%%%%%%%%%%%%%%%%%%%%%%%%%%%%%%%%%%%%%%%%%%%%%%%%%%%%%%%%%%%%%%%%%%%%%%%%
% Zacatek textu vlastniho protokolu
%%%%%%%%%%%%%%%%%%%%%%%%%%%%%%%%%%%%%%%%%%%%%%%%%%%%%%%%%%%%%%%%%%%%%%%%%%%%%%%
%%%%%%%%%%%%%%%%%%%%%%%%%%%%%%%%%%%%%%%%%%%%%%%%%%%%%%%%%%%%%%%%%%%%%%%%%%%%%%%


\section{Úvod}

    \paragraph{} V první části jsem měl určovat modul pružnosti drátu přímou metodou.
    Na drát toušťky $d$ se pustepně zavěšovala závaží a měřila se závislost prověšení 
    $\Delta l$ na celkové hmotnosti $m$.

    \begin{equation}
        \Delta l = \frac{4 g l}{\pi d^{2} E} m 
    \end{equation}

    Ze směrnice $k = \frac{4 g l}{\pi d^{2} E}$ pak můřu určit modul pružnosti $E$.

    \paragraph{} V druhé části jsem měřil modul pružnosti z průhybu nosníku. Ze změřené
    závislosti $y$ jsem určil modul pružnosti $E$. 

    \begin{equation}
        y = \frac{m g l^{3}}{4 E a^{3} b}
    \end{equation}

    Ze směrnice jsem potom podobně jako u předchozího experimentu získat samotnou
    hodnotu $E$.

    \paragraph{} V poslední části se pak měl určit modul pružnosti drátu ve smyku
    dynamickou metodou. Ze změřené periody kmitů $T$ lze modul pružnosti $G$ určit
    jako

    \begin{equation}
        G = \frac{16 \pi m R^{2} l}{5 r^{4} T^{2}}
    \end{equation}

\section{Výsledky}

    \subsection{Modul pružnosti v tahu měřený přímou metodou}

        \paragraph{} Z naměřených stačí spočítat lineární regresí sklon, ten
        je roven konstantě $k$.

\begin{lstlisting}[language=Python][h]
import numpy
import matplotlib.pyplot as plt
from scipy import stats

data = [float(i.strip()) for i in open('data_1').readlines()[:21]]
weights = [float(i.strip()) for i in open('vahy_1').readlines()[:10]]

x = list()
last = 0
for i in weights:
    last = last + i
    x.append(last)

x.insert(0, 0)

x_rev = list(reversed(x))

y = data[:11]
y_rev = data[10:]

slope, intercept, r_value, p_value, std_err = stats.linregress(x, y)
print(slope, std_err)
slope, intercept, r_value, p_value, std_err = stats.linregress(x_rev, y_rev)
print(slope, std_err)\end{lstlisting}

        Tento script výhodí směrnici i s odchylkou.

        $$k_{1} = \left(432 \pm 3\right) \cdot 10^{-6} m \cdot kg^{-1}$$
        $$k_{2} = \left(425 \pm 1\right) \cdot 10^{-6} m \cdot kg^{-1}$$

        tloušťka drátu je $d = 5.0 \cdot 10^{-4} \, m$ a výchozí délka drátu
        $l = 1.565 \, m$. Modul pružnosti lze pak určit pomocí rovnice

        \begin{equation}
            E = \frac{4 g l}{\pi d^{2} k}
        \end{equation}

        A odtud vychází modul pružnosti

        $$ E_{1} = \left(18 \pm 4\right) \cdot 10^{10} Pa$$
        $$ E_{2} = \left(18 \pm 4\right) \cdot 10^{10} Pa$$

    \subsection{Modul pružnosti v tahu z průhybu nosníku}

        \paragraph{} Při zpracování dat jsem postupoval podobně jako u předchozího
        měření, zjistil jsem směrnici $k$ a z ní jsem určil modul pružnosti $E$. 

\begin{lstlisting}[language=Python][h]
import numpy
import matplotlib.pyplot as plt
from scipy import stats

def is_number(n):
    try:
        float(n)
    except ValueError:
        return False
    return True

data_raw = [float(i.strip()) * 1e-3 for i in open('data_2').readlines() if is_number(i)]
weights = [float(i.strip()) * 1e-3 for i in open('vahy_2').readlines()[:10]]
sizes = [float(i.strip()) * 1e-3 for i in open('data_2_rozmery').readlines() if is_number(i)]

a, ua, b, ub = list(), list(), list(), list()

for i in range(5):
    a.append(numpy.average(sizes[20 * i : 20 * i + 10]))
    b.append(numpy.average(sizes[20 * i + 10 : 20 * i + 20]))

    ub.append(numpy.average((sizes[20 * i : 20 * i + 10] - a[-1])**2))
    ua.append(numpy.average((sizes[20 * i + 10 : 20 * i + 20] - b[-1])**2))

data_points = [(0, 9), (9, 30), (30, 51), (51, 72), (72, 93)]
data = list()
for i, j in data_points:
    data.append(data_raw[i:j])

_x = list()
_x.append(0)
last = 0
for i in weights:
    last = last + i
    _x.append(last)

x = [_x[:5]]
for _ in range(4):
    x.append(_x)


def analyze(x, y):
    x_rev = list(reversed(x))
    k1, _, _, _, err1 = stats.linregress(x, y[:len(x)])
    k2, _, _, _, err2 = stats.linregress(x_rev, y[len(x) - 1:])
    return (k1, err1, k2, err2)


def calculate_e(a, ua, b, ub, l, ul, k, uk):
    from math import pow, sqrt
    g = 9.81
    pow2 = lambda x: pow(x, 2)
    E = g * pow(l, 3) / (4 * k * pow(a, 3) * b)
    uE = g * pow(l, 3) / (4 * k * pow(a, 3) * b) * sqrt(pow2(3 * ua / a) + pow2(ub / b) + pow2(3 * ul / l) + pow2(uk / k))

    return (E, uE)


l, ul = (0.898, 0.0005)
for i in range(5):
    k1, uk1, k2, uk2 = analyze(x[i], data[i])
    print(calculate_e(a[i], ua[i], b[i], ub[i], l, ul, k1, uk1))
    print(calculate_e(a[i], ua[i], b[i], ub[i], l, ul, k2, uk2)) \end{lstlisting}

    výsledné moduly pružnosti jsou

    $$ E_{uhlik} = \left(9 \pm 1\right) \, GPa$$
    $$ E_{mosaz} = \left(4603 \pm 12\right) \, GPa$$
    $$ E_{ocel} = \left(8372 \pm 1\right) \, GPa$$
    $$ E_{hlinik} = \left(12108 \pm 25\right) \, GPa$$
    $$ E_{mosaz}^{(2)} = \left(96.9 \pm 0.2\right) \, GPa$$

    \subsection{Modul pružnosti ve smyku}

        \paragraph{} Následujícím scriptem jsem určil hodnty neznámých veličin a
        jejich nejistoty, script nakonec vypočítá výsledný modul pružnosti.

\begin{lstlisting}[language=Python][h]
import numpy
from math import pi, sqrt, pow

m = 5.905
um = 0.0005
_T = [3.96, 3.99, 4.07, 3.99, 3.99, 4.07, 4.02, 3.99, 3.96, 3.91] # s
_R = [9.576, 9.562, 9.562, 9.572, 9.560, 9.524, 9.570, 9.578, 9.514, 9.568] # cm
_l = [51.5, 51.4, 51.6, 51.4, 51.5, 51.5, 51.4, 51.5, 51.4, 51.6] # cm
_r = [1.00, 1.00, 0.99, 1.00, 1.00, 0.99, 0.99, 1.00, 0.99, 1.00] # mm

# uprava na zakladni jednotky
_R = [i * 1e-2 for i in _R]
_l = [i * 1e-2 for i in _l]
_r = [i * 1e-3 for i in _r]

T = numpy.average(_T)
R = numpy.average(_R)
l = numpy.average(_l)
r = numpy.average(_r)

uT = numpy.average((_T - T)**2)
uR = numpy.average((_R - R)**2)
ul = numpy.average((_l - l)**2)
ur = numpy.average((_r - r)**2)

pow2 = lambda x: pow(x, 2)
G = 16 * pi * m * pow(R, 2) * l / (5 * pow(r, 4) * pow(T, 2))
uG = G * sqrt(pow2(um /m) + pow2(2 * uR / R) + pow2(ul / l) + pow2(4 * ur / r) + pow2(2 * uT / T))

print(G, uG) \end{lstlisting}

    $$ G = \left(1778 \pm 2\right) \cdot 10^{7} \, Pa$$

\section{Zhodnocení měření, závěr}

    \paragraph{} U měřeních, kde bylo potřeba měřit hmotnosti válečků, je zajímavé, že
    přesto, že jednotlivé hmotnosti se liší nepatrně, celková hmotnost by vyrobila rozdíl
    několik gramů. Výsledky se vůči tabulkovým hodnotám někdy liší docela výrazně, zejména hodnoty
    modulu pružnosti v tahu pohybující se v tisisích $GPa$ jsou pravděpodobně zatíženy nějakou
    chybou. Zbytek se alespoň řádově shoduje.

\end{document}

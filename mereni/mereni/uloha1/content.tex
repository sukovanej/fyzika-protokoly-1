% Hlavicka pro protokoly z fyzikalniho praktika.
% Verze pro: LaTeX
% Verze hlavicky: 22. 2. 2007
% Autor: Ustav fyziky kondenzovanych latek
% Ke stazeni: www.physics.muni.cz/ufkl/Vyuka/
% Licence: volne k pouziti, nejlepe k vcasnemu odevzdani protokolu z Vaseho mereni.

\documentclass[a4paper,11pt]{article}

% Kodovani (cestiny) v dokumentu: cp1250
% \usepackage[cp1250]{inputenc}	% Omezena stredoevropska kodova stranka, pouze MSW.
\usepackage[utf8]{inputenc}	% Doporucujeme pouzivat UTF-8 (unicode).

%%% Nemente:
\usepackage[margin=2cm]{geometry}
\newtoks\jmenopraktika \newtoks\jmeno \newtoks\datum
\newtoks\obor \newtoks\skupina \newtoks\rocnik \newtoks\semestr
\newtoks\cisloulohy \newtoks\jmenoulohy
\newtoks\tlak \newtoks\teplota \newtoks\vlhkost
%%% Nemente - konec.


%%%%%%%%%%% Doplnte pozadovane polozky:

\jmenopraktika={Fyzikální praktikum 1}  % nahradte jmenem vaseho predmetu
\jmeno={Milan Suk}            % nahradte jmenem mericiho
\datum={26. února 2018}        % nahradte datem mereni ulohy
\obor={F}                     % nahradte zkratkou vami studovaneho oboru
\skupina={PO 8:00}            % nahradte dobou vyuky vasi seminarni skupiny
\rocnik={I}                  % nahradte rocnikem, ve kterem studujete
\semestr={II}                 % nahradte semestrem, ve kterem studujete

\cisloulohy={1}               % nahradte cislem merene ulohy
\jmenoulohy={Měření hustosty válečku} % nahradte jmenem merene ulohy

\tlak={97,9}                   % nahradte tlakem pri mereni (v hPa)
\teplota={21,4}               % nahradte teplotou pri mereni (ve stupnich Celsia)
\vlhkost={40}               % nahradte vlhkosti vzduchu pri mereni (v %)

%%%%%%%%%%% Konec pozadovanych polozek.


%%%%%%%%%%% Uzitecne balicky:
\usepackage[czech]{babel}
\usepackage{graphicx}
\usepackage{amsmath}
\usepackage{xspace}
\usepackage{url}
\usepackage{indentfirst}
\usepackage{listings}
\usepackage{color}


\definecolor{codegreen}{rgb}{0,0.6,0}
\definecolor{codegray}{rgb}{0.5,0.5,0.5}
\definecolor{codepurple}{rgb}{0.58,0,0.82}
\definecolor{backcolour}{rgb}{0.95,0.95,0.92}
 
\lstdefinestyle{mystyle}{
    backgroundcolor=\color{backcolour},   
    commentstyle=\color{codegreen},
    keywordstyle=\color{magenta},
    numberstyle=\tiny\color{codegray},
    stringstyle=\color{codepurple},
    basicstyle=\footnotesize,
    breakatwhitespace=false,         
    breaklines=true,                 
    captionpos=b,                    
    keepspaces=true,                 
    numbers=left,                    
    numbersep=5pt,                  
    showspaces=false,                
    showstringspaces=false,
    showtabs=false,                  
    tabsize=2
}
 
\lstset{style=mystyle}

%%%%%% Zamezeni parchantu:
\widowpenalty 10000 \clubpenalty 10000 \displaywidowpenalty 10000
%%%%%% Parametry pro moznost vsazeni vetsiho poctu obrazku na stranku
\setcounter{topnumber}{3}	  % max. pocet floatu nahore (specifikace t)
\setcounter{bottomnumber}{3}	  % max. pocet floatu dole (specifikace b)
\setcounter{totalnumber}{6}	  % max. pocet floatu na strance celkem
\renewcommand\topfraction{0.9}	  % max podil stranky pro floaty nahore
\renewcommand\bottomfraction{0.9} % max podil stranky pro floaty dole
\renewcommand\textfraction{0.1}	  % min podil stranky, ktery musi obsahovat text
\intextsep=8mm \textfloatsep=8mm  %\intextsep pro ulozeni [h] floatu a \textfloatsep pro [b] or [t]

% Tecky za cisly sekci:
\renewcommand{\thesection}{\arabic{section}.}
\renewcommand{\thesubsection}{\thesection\arabic{subsection}.}
% Jednopismenna mezera mezi cislem a nazvem kapitoly:
\makeatletter \def\@seccntformat#1{\csname the#1\endcsname\hspace{1ex}} \makeatother


%%%%%%%%%%%%%%%%%%%%%%%%%%%%%%%%%%%%%%%%%%%%%%%%%%%%%%%%%%%%%%%%%%%%%%%%%%%%%%%
%%%%%%%%%%%%%%%%%%%%%%%%%%%%%%%%%%%%%%%%%%%%%%%%%%%%%%%%%%%%%%%%%%%%%%%%%%%%%%%
% Zacatek dokumentu
%%%%%%%%%%%%%%%%%%%%%%%%%%%%%%%%%%%%%%%%%%%%%%%%%%%%%%%%%%%%%%%%%%%%%%%%%%%%%%%
%%%%%%%%%%%%%%%%%%%%%%%%%%%%%%%%%%%%%%%%%%%%%%%%%%%%%%%%%%%%%%%%%%%%%%%%%%%%%%%

\begin{document}

%%%%%%%%%%%%%%%%%%%%%%%%%%%%%%%%%%%%%%%%%%%%%%%%%%%%%%%%%%%%%%%%%%%%%%%%%%%%%%%
% Nemente:
%%%%%%%%%%%%%%%%%%%%%%%%%%%%%%%%%%%%%%%%%%%%%%%%%%%%%%%%%%%%%%%%%%%%%%%%%%%%%%%
\thispagestyle{empty}

{
\begin{center}
\sf 
{\Large Ústav fyzikální elektroniky Přírodovědecké fakulty Masarykovy univerzity} \\
\bigskip
{\huge \bfseries FYZIKÁLNÍ PRAKTIKUM} \\
\bigskip
{\Large \the\jmenopraktika}
\end{center}

\bigskip

\sf
\noindent
\setlength{\arrayrulewidth}{1pt}
\begin{tabular*}{\textwidth}{@{\extracolsep{\fill}} l l}
\large {\bfseries Zpracoval:}  \the\jmeno & \large  {\bfseries Naměřeno:} \the\datum\\[2mm]
\large  {\bfseries Obor:} \the\obor  \hspace{40mm}  {\bfseries Skupina:} \the\skupina %
&\large {\bfseries Testováno:}\\
\\
\hline
\end{tabular*}
}

\bigskip

{
\sf
\noindent \begin{tabular}{p{3cm} p{0.6\textwidth}}
\Large  Úloha č. {\bfseries \the\cisloulohy:} \par
&\Large \bfseries \the\jmenoulohy  \\[2mm]
\end{tabular}
}

%%%%%%%%%%%%%%%%%%%%%%%%%%%%%%%%%%%%%%%%%%%%%%%%%%%%%%%%%%%%%%%%%%%%%%%%%%%%%%%
% konec Nemente.
%%%%%%%%%%%%%%%%%%%%%%%%%%%%%%%%%%%%%%%%%%%%%%%%%%%%%%%%%%%%%%%%%%%%%%%%%%%%%%%

%%%%%%%%%%%%%%%%%%%%%%%%%%%%%%%%%%%%%%%%%%%%%%%%%%%%%%%%%%%%%%%%%%%%%%%%%%%%%%%
%%%%%%%%%%%%%%%%%%%%%%%%%%%%%%%%%%%%%%%%%%%%%%%%%%%%%%%%%%%%%%%%%%%%%%%%%%%%%%%
% Zacatek textu vlastniho protokolu
%%%%%%%%%%%%%%%%%%%%%%%%%%%%%%%%%%%%%%%%%%%%%%%%%%%%%%%%%%%%%%%%%%%%%%%%%%%%%%%
%%%%%%%%%%%%%%%%%%%%%%%%%%%%%%%%%%%%%%%%%%%%%%%%%%%%%%%%%%%%%%%%%%%%%%%%%%%%%%%


\section{Úvod}

    \paragraph{} Cílem toho měření je zjistit hustotu válečku s válcovým výřezem. 
    Hustota má být vypočtena pomocí měření jeho výšky, vnitřního a vnějšího průměru, a
    jeho hmotnosti.

    \paragraph{} Objem válečku určím pomocí změřené výšky $h$, vnitřního poloměru $r$ a
    vnějšího poloměr $R$ pomocí rovnice

    \begin{equation}
    V = h \pi (R^{2} - r^{2})
    \end{equation}

    \paragraph{} Pak se změřenou hmotností $m$ lze určit hustotu válečku jako

    \begin{equation}
    \rho = \frac{m}{h \pi (R^{2} - r^{2})}
    \end{equation}

    při vlastním měření jsem ovšem ve skutečnosti měřil průměry $2R \to R$ a
    $2r \to r$. Výslednou hustotu určím podle upraveného vztahu

    \begin{equation}
    \rho = \frac{4m}{h \pi (R^{2} - r^{2})}
    \end{equation}

    \subsection{Zpracování chyb měření}

        \paragraph{} Nejistotu měření určím ze \textit{Zákona šíření nejistoty} jako

        \begin{equation}
        u(\rho) = \sqrt{
          \left(\frac{\partial \rho}{\partial m}\right)^{2} \cdot u(m)^{2}
        + \left(\frac{\partial \rho}{\partial r}\right)^{2} \cdot u(r)^{2} 
        + \left(\frac{\partial \rho}{\partial R}\right)^{2} \cdot u(R)^{2}
        + \left(\frac{\partial \rho}{\partial h}\right)^{2} \cdot u(h)^{2}
        }
        \end{equation}

        konkrétně potom

        \begin{equation}
        u(\rho) = \sqrt{
              \left(\frac{4u(m)}{h \pi \left(R^{2} - r^{2}\right)}\right)^{2}
            + \left(\frac{8mR \cdot u(R)}{h \pi \left(R^{2} - r^{2}\right)^{2}}\right)^{2}
            + \left(\frac{8mr \cdot u(r)}{h \pi \left(R^{2} - r^{2}\right)^{2}}\right)^{2}
            + \left(\frac{4m \cdot u(h)}{h^{2} \pi \left(R^{2} - r^{2}\right)}\right)^{2}
        }
        \end{equation}

\section{Postup měření}

    \paragraph{} Pomocí posuvného měřidla (s přesností $0.02\,mm$) jsem 
    nejdříve změřil vnitřní průměr $r$ a vnější průměr válečku $R$. Poté jsem
    pomocí mikrometru (s přesností $0.01\,mm$) změřil výšku válečku $h$ a 
    nakonec pomocí laboratorních vah jsem určil hmotnost válečku $m$.

\section{Výsledky}

    \subsection{Měření průměrů a výšky válce}

     \begin{table}[h]
        \centering
            \begin{tabular}{ | l | l | l | }
                \hline
                $R$ $[mm]$ & $r$ $[mm]$& $h$ $[mm]$ \\ \hline
                46.06 & 9.80 & 14.83 \\ \hline
                46.10 & 9.76 & 14.75 \\ \hline
                46.06 & 9.76 & 14.76 \\ \hline
                46.06 & 9.78 & 14.77 \\ \hline
                46.08 & 9.76 & 14.81 \\ \hline
                46.06 & 9.82 & 14.82 \\ \hline
                46.06 & 9.72 & 14.81 \\ \hline
                46.06 & 9.78 & 14.86 \\ \hline
                46.10 & 9.80 & 14.83 \\ \hline
                46.08 & 9.80 & 14.77 \\
                \hline
            \end{tabular}
        \caption{Měření průměrů a výšky válce}
        \label{fig:method_b}
    \end{table}

    Hmotnost válečku jsem měřil na laboratorních vahách.

    $$\overline{m} = 159.096g$$

    Hodnoty průměrů a odchylek si nechám vypočítat počítačem.

    \begin{lstlisting}[language=Python]
import statistics
import math

R = [46.06, 46.10, 46.06, 46.06, 46.08, 46.06, 46.06, 46.06, 46.10, 46.08]
r = [9.80, 9.76, 9.78, 9.76, 9.82, 9.82, 9.72, 9.78, 9.80, 9.80]
h = [14.83, 14.75, 14.76, 14.77, 14.81, 14.82, 14.81, 14.86, 14.83, 14.77]

result = {}

student_9_6827 = 1.067
student_9_9973 = 4.094

for key, data in {"R": R, "r": r, "h": h}.items():
    mean = statistics.mean(data)
    stdev = statistics.stdev(data)
    stdev1 = student_9_6827 * stdev
    u = stdev1 / math.sqrt(len(data))
    r = stdev1 / mean * 100

    result[key] = {
        "mean": mean,
        "u": u,
        "r": r
    }

print(result)\end{lstlisting} 

    \paragraph{} Provoláním tohoto scriptu získám následující výstup

    \begin{lstlisting}[language=Bash]
> python3 valecek.py | sed -e "s/'/\"/g" | jq 
{
  "R": {
    "mean": 46.072,
    "u": 0.005690666666666419,
    "r": 0.03905944623950355
  },
  "r": {
    "mean": 9.784,
    "u": 0.01045442222219864,
    "r": 0.33789642112864493
  },
  "h": {
    "mean": 14.801,
    "u": 0.01226405507072511,
    "r": 0.26202518325267715
  }
} \end{lstlisting}

    \paragraph{} Spolu s kombinovanými nejistotami $u = \sqrt{u_{A}^{2} +
    u_{B}^{2}}$ získám následující údaje o změřených veličinách.

    $$m = (191.5761 \pm 0.001)\,g$$
    $$R = (46.07 \pm 0.01)\,mm$$
    $$r = (9.78 \pm 0.01)\,mm$$
    $$h = (14.80 \pm 0.01)\,mm$$

    \begin{lstlisting}[language=Python]
from math import pow, sqrt, pi

pow2 = lambda x: pow(x, 2)

R = 46.07 / 1000
r = 9.78 / 1000
h = 14.80 / 1000
m = 191.5761 / 1000

uR = 0.01 / 1000
ur = 0.01 / 1000
uh = 0.01 / 1000
um = 0.00005 / 1000

u = 4 / (pi * h * (pow2(R) - pow2(r))) * \
        sqrt(pow2(um) + pow2(m * uh / h) + pow2(2 * R * m * uR / (pow2(R) - \
        pow2(r))) + pow2(2 * r * m * ur / (pow2(R) - pow2(r))))

rho = m / (pi * h * (pow2(R / 2) - pow2(r / 2)))

print("({} +- {}) kg * m^-3".format(rho, u)) \end{lstlisting}

    A provoláním získám celkový výsledek.

    \begin{lstlisting}[language=Bash]
> python3 valecek_nejistota.py
(8131.666998920163 +- 6.668565171779014) kg * m^-3 \end{lstlisting}

    \paragraph{} Hledaná hustota tedy je
    $$\rho = \left(8132 \pm 7 \right)\,kg \cdot m^{-3}$$.

\section{Zhodnocení měření, závěr}

    \paragraph{} Podle zjištěné hustoty byl neznámý materiál 
    pravděpodobně \textbf{mosaz}.

\end{document}

% Hlavicka pro protokoly z fyzikalniho praktika.
% Verze pro: LaTeX
% Verze hlavicky: 22. 2. 2007
% Autor: Ustav fyziky kondenzovanych latek
% Ke stazeni: www.physics.muni.cz/ufkl/Vyuka/
% Licence: volne k pouziti, nejlepe k vcasnemu odevzdani protokolu z Vaseho mereni.

\documentclass[a4paper,11pt]{article}

% Kodovani (cestiny) v dokumentu: cp1250
% \usepackage[cp1250]{inputenc}	% Omezena stredoevropska kodova stranka, pouze MSW.
\usepackage[utf8]{inputenc}	% Doporucujeme pouzivat UTF-8 (unicode).
\usepackage{subfig}
\usepackage{float}

\usepackage{xparse}

\NewDocumentCommand{\codeword}{v}{%
\texttt{\textcolor{codepurple}{#1}}%
}

%%% Nemente:
\usepackage[margin=2cm]{geometry}
\newtoks\jmenopraktika \newtoks\jmeno \newtoks\datum
\newtoks\obor \newtoks\skupina \newtoks\rocnik \newtoks\semestr
\newtoks\cisloulohy \newtoks\jmenoulohy
\newtoks\tlak \newtoks\teplota \newtoks\vlhkost
%%% Nemente - konec.


%%%%%%%%%%% Doplnte pozadovane polozky:

\jmenopraktika={Fyzikální praktikum 1}  % nahradte jmenem vaseho predmetu
\jmeno={Milan Suk}            % nahradte jmenem mericiho
\datum={3. dubna 2018}        % nahradte datem mereni ulohy
\obor={F}                     % nahradte zkratkou vami studovaneho oboru
\skupina={PO 8:00}            % nahradte dobou vyuky vasi seminarni skupiny
\rocnik={I}                  % nahradte rocnikem, ve kterem studujete
\semestr={II}                 % nahradte semestrem, ve kterem studujete

\cisloulohy={7}               % nahradte cislem merene ulohy
\jmenoulohy={Měření poissonovy konstanty vzduchu} % nahradte jmenem merene ulohy

\tlak={97,9}                   % nahradte tlakem pri mereni (v hPa)
\teplota={21,4}               % nahradte teplotou pri mereni (ve stupnich Celsia)
\vlhkost={40}               % nahradte vlhkosti vzduchu pri mereni (v %)

%%%%%%%%%%% Konec pozadovanych polozek.


%%%%%%%%%%% Uzitecne balicky:
\usepackage[czech]{babel}
\usepackage{graphicx}
\usepackage{amsmath}
\usepackage{xspace}
\usepackage{url}
\usepackage{indentfirst}
\usepackage{listings}
\usepackage{color}


\definecolor{codegreen}{rgb}{0,0.6,0}
\definecolor{codegray}{rgb}{0.5,0.5,0.5}
\definecolor{codepurple}{rgb}{0.58,0,0.82}
\definecolor{backcolour}{rgb}{0.95,0.95,0.92}
 
\lstdefinestyle{mystyle}{
    backgroundcolor=\color{backcolour},   
    commentstyle=\color{codegreen},
    keywordstyle=\color{magenta},
    numberstyle=\tiny\color{codegray},
    stringstyle=\color{codepurple},
    basicstyle=\footnotesize,
    breakatwhitespace=false,         
    breaklines=true,                 
    captionpos=b,                    
    keepspaces=true,                 
    numbers=left,                    
    numbersep=5pt,                  
    showspaces=false,                
    showstringspaces=false,
    showtabs=false,                  
    tabsize=2
}
 
\lstset{style=mystyle}

%%%%%% Zamezeni parchantu:
\widowpenalty 10000 \clubpenalty 10000 \displaywidowpenalty 10000
%%%%%% Parametry pro moznost vsazeni vetsiho poctu obrazku na stranku
\setcounter{topnumber}{3}	  % max. pocet floatu nahore (specifikace t)
\setcounter{bottomnumber}{3}	  % max. pocet floatu dole (specifikace b)
\setcounter{totalnumber}{6}	  % max. pocet floatu na strance celkem
\renewcommand\topfraction{0.9}	  % max podil stranky pro floaty nahore
\renewcommand\bottomfraction{0.9} % max podil stranky pro floaty dole
\renewcommand\textfraction{0.1}	  % min podil stranky, ktery musi obsahovat text
\intextsep=8mm \textfloatsep=8mm  %\intextsep pro ulozeni [h] floatu a \textfloatsep pro [b] or [t]

% Tecky za cisly sekci:
\renewcommand{\thesection}{\arabic{section}.}
\renewcommand{\thesubsection}{\thesection\arabic{subsection}.}
% Jednopismenna mezera mezi cislem a nazvem kapitoly:
\makeatletter \def\@seccntformat#1{\csname the#1\endcsname\hspace{1ex}} \makeatother


%%%%%%%%%%%%%%%%%%%%%%%%%%%%%%%%%%%%%%%%%%%%%%%%%%%%%%%%%%%%%%%%%%%%%%%%%%%%%%%
%%%%%%%%%%%%%%%%%%%%%%%%%%%%%%%%%%%%%%%%%%%%%%%%%%%%%%%%%%%%%%%%%%%%%%%%%%%%%%%
% Zacatek dokumentu
%%%%%%%%%%%%%%%%%%%%%%%%%%%%%%%%%%%%%%%%%%%%%%%%%%%%%%%%%%%%%%%%%%%%%%%%%%%%%%%
%%%%%%%%%%%%%%%%%%%%%%%%%%%%%%%%%%%%%%%%%%%%%%%%%%%%%%%%%%%%%%%%%%%%%%%%%%%%%%%

\begin{document}

%%%%%%%%%%%%%%%%%%%%%%%%%%%%%%%%%%%%%%%%%%%%%%%%%%%%%%%%%%%%%%%%%%%%%%%%%%%%%%%
% Nemente:
%%%%%%%%%%%%%%%%%%%%%%%%%%%%%%%%%%%%%%%%%%%%%%%%%%%%%%%%%%%%%%%%%%%%%%%%%%%%%%%
\thispagestyle{empty}

{
\begin{center}
\sf 
{\Large Ústav fyzikální elektroniky Přírodovědecké fakulty Masarykovy univerzity} \\
\bigskip
{\huge \bfseries FYZIKÁLNÍ PRAKTIKUM} \\
\bigskip
{\Large \the\jmenopraktika}
\end{center}

\bigskip

\sf
\noindent
\setlength{\arrayrulewidth}{1pt}
\begin{tabular*}{\textwidth}{@{\extracolsep{\fill}} l l}
\large {\bfseries Zpracoval:}  \the\jmeno & \large  {\bfseries Naměřeno:} \the\datum\\[2mm]
\large  {\bfseries Obor:} \the\obor  \hspace{40mm}  {\bfseries Skupina:} \the\skupina %
&\large {\bfseries Testováno:}\\
\\
\hline
\end{tabular*}
}

\bigskip

{
\sf
\noindent \begin{tabular}{p{3cm} p{0.6\textwidth}}
\Large  Úloha č. {\bfseries \the\cisloulohy:} \par
&\Large \bfseries \the\jmenoulohy  \\[2mm]
\end{tabular}
}

%%%%%%%%%%%%%%%%%%%%%%%%%%%%%%%%%%%%%%%%%%%%%%%%%%%%%%%%%%%%%%%%%%%%%%%%%%%%%%%
% konec Nemente.
%%%%%%%%%%%%%%%%%%%%%%%%%%%%%%%%%%%%%%%%%%%%%%%%%%%%%%%%%%%%%%%%%%%%%%%%%%%%%%%

%%%%%%%%%%%%%%%%%%%%%%%%%%%%%%%%%%%%%%%%%%%%%%%%%%%%%%%%%%%%%%%%%%%%%%%%%%%%%%%
%%%%%%%%%%%%%%%%%%%%%%%%%%%%%%%%%%%%%%%%%%%%%%%%%%%%%%%%%%%%%%%%%%%%%%%%%%%%%%%
% Zacatek textu vlastniho protokolu
%%%%%%%%%%%%%%%%%%%%%%%%%%%%%%%%%%%%%%%%%%%%%%%%%%%%%%%%%%%%%%%%%%%%%%%%%%%%%%%
%%%%%%%%%%%%%%%%%%%%%%%%%%%%%%%%%%%%%%%%%%%%%%%%%%%%%%%%%%%%%%%%%%%%%%%%%%%%%%%


\section{Úvod}

    \paragraph{} V první částí jsem měřil hodnotu Poissonovy konstanty 
    Clément-Desormesovou metodou. Do nádoby se po otevření ventilu natlakuje vzduch.
    hladina se ustálí na výšce $h_{1}$, která se zaznamená. Pak se rychle otevře a
    zavře ventil a po ustálení se zaznamená výška $h_{2}$. Pro Poissonovu
    konstantu v těchto podmínkách platí

    \begin{equation}
        \kappa = \frac{h_{1}}{h_{1} - h_{2}}
    \end{equation}

    \paragraph{} V druhé části se měla Poissonova konstanta určit z rychlosti 
    šíření vzvuku v plynu. Na generátoru jsem nastavil frekvenci $f$, posouval 
    pástem v trubici a zaznamenával polohy $x_{i}$ maxim. Jejich rozdílem 
    $x_{i + 1} - x_{i}$ zjistím vlnovou délku $\lambda$. Při známém tlaku $p$
    a hustotě prostředí $\rho$ platí pro Poissonovu konstantu

    \begin{equation}
        \kappa = \frac{c^{2} \rho}{p}
    \end{equation}

    
\section{Postup měření}

    \subsection{Clément-Desormesova trubice}

        \paragraph{} Po změření stačí určit průměrnou hodnotu přes všechny
        získané hodnoty.

\begin{lstlisting}[language=Bash][H]
import numpy
h1 = [53, 68, 74, 107, 108, 123, 129, 154, 184, 193]
h2 = [1, 6, 7, 15, 16, 21, 23, 27, 34, 36]

kappa_list = list()

for i in range(len(h1)):
    kappa_list.append(h1[i] / (h1[i] - h2[i]))

kappa = numpy.average(kappa_list)
kappa_err = numpy.average((kappa_list - kappa)**2)

print(f"{kappa} +- {kappa_err}")\end{lstlisting}

        výsledek vyhodnocení je

        $$ \kappa = 1.165 \pm 0.004$$

    \subsection{Kundtova trubice}

        \paragraph{} Následující python script projdu získané body $x_{i}$, určí jejich rozdíly
        a dvojnásobek vrátí jako hodnotu $\lambda$, následně dosadí do vytahu (2), čímž 
        určí hodnotu Poissonovy konstanty.

\begin{lstlisting}[language=Bash][H]
import numpy
from math import pow, sqrt

pow2 = lambda x: x * x

data = [
    (961.7, [24.0, 41.7, 59.6, 77.5, 95.5]),
    (1255.1, [33.0, 46.7, 60.2, 74., 87.8, 101.5]),
    (1500.6, [24.3, 36.0, 47.5, 58.6, 70.2, 81.6, 93.0, 104.5]),
    (1765.4, [29.1, 39.0, 48.5, 58.4, 68.0, 77.8, 87.5, 97.3, 107.0]),
    (2012.1, [22.0, 30.8, 40.5, 49.2, 57.6, 65.8, 74.6, 83.2, 91.8, 100.3, 108.9])
]

rho = 1.129
p, up = 96650, 50
uf = 0.1

def calc_lambda(x: list):
    last = x[0]
    l = list()

    for i in range(1, len(x)):
        l.append((x[i] - last) * 1e-2) # cm -> m
        last = x[i]

    x_average = numpy.average(l)
    x_err = numpy.average((l - x_average)**2)

    return (2 * x_average, 2 * x_err)

def calc_kappa(lam, f):
    kappa = pow(lam[0] * f, 2) * rho / p
    kappa_err = kappa * sqrt(pow2(2 * uf / f) + pow2(2 * lam[1] / lam[0]) + pow2(up / p))

    return (round(kappa, 3), round(kappa_err, 3))

for d in data:
    lam = calc_lambda(d[1])
    print(calc_kappa(lam, d[0])) \end{lstlisting}

    $$ \kappa_{1} = \left(1.381 \pm 0.001\right)$$
    $$ \kappa_{2} = \left(1.381 \pm 0.001\right)$$
    $$ \kappa_{3} = \left(1.381 \pm 0.001\right)$$
    $$ \kappa_{4} = \left(1.381 \pm 0.001\right)$$
    $$ \kappa_{5} = \left(1.429 \pm 0.001\right)$$

\section{Výsledky}

    \paragraph{} První metodou vyšla hodnota $\kappa$ asi $1.165$.  Výsledky Poissonovy 
    konstanty měřené Kundtovou trubicí vyšli až na poslední téměř stejně $1.381$. Hodnoty
    se sice poměrně liší, ale alespoň řádově jsem dostal poměrně dobrý výsledek.
\end{document}
